\documentclass[11pt,a4paper]{article}

% Apenas pacotes essenciais
\usepackage[margin=2.5cm]{geometry}
\usepackage{setspace}

% Bibliografia com biblatex - versão simples
\usepackage[backend=biber,style=authoryear,sorting=nyt]{biblatex}
\addbibresource{reference.bib}

% Configurações básicas
\onehalfspacing
\setlength{\parindent}{1.5cm}

% Informações do artigo
\title{\textbf{Neonatal Hypothermia and Neonatal Anoxia}}

\author{
	\textbf{iTec/FURG}$^{1}$ \\
	\textbf{Calvimontes S. C.}$^{2}$ \\
	\textbf{LLaMA 3.3,}$^{3}$ \\
	\\
	\small $^{1}$Centro de Robótica e Ciência de Dados \\
	\small Universidade Federal do Rio Grande (unidade EMbrapii / FURG) \\
	\small Rio Grande do Sul (Brazil) \\
	\\
	\small $^{2}$Universidade Federal Fluminense \\
	\small Email: sergiocc@id.uff.br \\
	\\
	\small $^{3}$Meta AI / Meta Platforms, Inc. - Menlo Park, Califórnia (EUA) \\
	\small Web site: https://www.meta.ai/
}

\date{\today}

\begin{document}
	
	\maketitle
	
	\newpage
	
	\section{Introduction}
	Therapeutic hypothermia is a neuroprotective strategy that reduces mortality and disability in newborns with hypoxic-ischemic encephalopathy due to perinatal asphyxia. The therapy should start within the first six hours after birth and consists of reducing the body temperature of neonates to an average temperature of 33°C to 34°C for 72 hours [4,6,7].
		
	\subsection{Research Objectives}
	This study reports two clinical cases describing the effects of neonatal hypothermia in babies with perinatal asphyxia and motor development in a follow-up program after hospital discharge.
	
	\section{Methods}
	This is a retrospective case report involving two children diagnosed with hypoxic-ischemic encephalopathy due to neonatal asphyxia and submitted to a hypothermia protocol in the Neonatal Intensive Care Unit (NICU).
	
	\subsection{Case Description}
	\subsubsection{Case 1}
	A newborn, born by cesarean section at 37 weeks of gestational age, weighing 3.055g and length of 46.5cm, presented with an Apgar score of 5 and 6 at 1 and 5 minutes, respectively. The patient had respiratory distress and was intubated.
		
	\section{Conclusion}
	The cases presented involved two children diagnosed with hypoxic-ischemic encephalopathy due to perinatal asphyxia that received a therapeutic protocol of hypothermia for 74 hours with strict monitoring of body temperature. They were followed up at the outpatient clinic by a multidisciplinary team, and in the assessment of motor development, it was observed that both patients had normal motor development.
	
	\newpage
	
	% Bibliografia automática
	\printbibliography[title=References]
		
\end{document}