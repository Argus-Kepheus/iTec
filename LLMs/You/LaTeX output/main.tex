\documentclass[11pt,a4paper]{article}

% Apenas pacotes essenciais
\usepackage[margin=2.5cm]{geometry}
\usepackage{setspace}

% Bibliografia com biblatex - versão simples
\usepackage[backend=biber,style=authoryear,sorting=nyt]{biblatex}
\addbibresource{reference.bib}

% Configurações básicas
\onehalfspacing
\setlength{\parindent}{1.5cm}

% Informações do artigo
\title{\textbf{NEONATAL HYPOTHERMIA AND NEONATAL ANOXIA}}

\author{
	\textbf{iTec/FURG}$^{1}$ \\
	\textbf{Calvimontes S. C.}$^{2}$ \\
	\textbf{YouChat}$^{3}$ \\
	\\
	\small $^{1}$Centro de Robótica e Ciência de Dados \\
	\small Universidade Federal do Rio Grande (unidade EMbrapii / FURG) \\
	\small Rio Grande do Sul (Brazil) \\
	\\
	\small $^{2}$Universidade Federal Fluminense \\
	\small Email: sergiocc@id.uff.br \\
	\\
	\small $^{3}$You.com - Palo Alto, Califórnia (EUA) \\
	\small Web site: https://you.com/
}

\date{\today}

\begin{document}
	
	\maketitle
	
	\newpage
	
	\section{Introduction}
	Therapeutic hypothermia is a neuroprotective strategy that reduces mortality and disability in newborns with hypoxic-ischemic encephalopathy due to perinatal asphyxia. The therapy should start within the first six hours after birth and consists of reducing the neonate's body temperature to an average of 33–34°C for 72 hours [\cite{Azzopardi2014, Thayyil2021, Abate2021}]. Hypothermia reduces brain metabolism by approximately 5\% for every 1°C decrease in body temperature, which delays the onset of cellular anoxic depolarization [\cite{Silveira2015}].
		
	\subsection{Research Objectives}
	This study reports two clinical cases describing the effects of neonatal hypothermia in infants with perinatal asphyxia and their motor development during follow-up after hospital discharge.
	
	\section{Methods}
	This retrospective case report involves two children diagnosed with hypoxic-ischemic encephalopathy due to neonatal asphyxia who underwent a hypothermia protocol in the Neonatal Intensive Care Unit (NICU). Data regarding the prenatal, perinatal, and postnatal periods were collected from the children's medical records. Subsequently, an interview with the guardian was conducted using a semi-structured maternal history guide, including general information about the mother and baby. The children were followed up in the high-risk outpatient clinic and evaluated using the Hammersmith Infant Neurological Examination (HINE), motor development assessment with the Alberta Infant Motor Scale (AIMS), and the Denver II screening test. The instruments were administered according to the assessment manuals' recommendations by trained evaluators. The study was approved by the University's Research Ethics Committee.
	
	\subsection{Case Description}
	\subsubsection{Case 1}
	A female newborn was delivered by cesarean section at 37 weeks of gestational age, weighing 3,055 g and measuring 46.5 cm in length. The patient had Apgar scores of 5 and 6 at the first and fifth minutes, respectively, requiring a cycle of positive pressure ventilation (PPV). The infant developed respiratory distress; thus, 20\% oxygen was delivered for one hour, followed by three hours of continuous positive airway pressure (CPAP). After four hours of life, the patient’s respiratory distress worsened, and cyanosis appeared in the extremities. She was intubated and, during intubation, presented an episode of upper limb hyperextension, internal rotation of the wrists, and seizure.
	Due to tests indicating perinatal asphyxia, the therapeutic hypothermia protocol was initiated by turning off the incubator until the patient reached the target temperature of 32–35°C, monitored every 20 minutes, and maintained for 74 hours. The infant was diagnosed with late neonatal sepsis in the NICU and required six days of antibiotic treatment. Transfontanellar ultrasound showed reduced sulci and diffuse hyperechogenicity. After seven days, cranial magnetic resonance imaging (MRI) demonstrated sequelae of a severe perinatal hypoxic-ischemic event. The patient remained 12 days in the NICU and 10 days in the ward, discharged on a mixed diet of breast milk and formula.
	At discharge, neurological examination revealed mild generalized hypotonia and symmetrical primitive reflexes (search reflex, palmar and plantar grasp, Moro, and tonic neck reflexes). Currently, at 3 years and 3 months of age, evaluations by the physiotherapy team at the pediatric outpatient clinic indicate motor development within the normal range for age.
		
	\section{Conclusion}
	These cases involved two children diagnosed with hypoxic-ischemic encephalopathy due to perinatal asphyxia who underwent a therapeutic hypothermia protocol for 74 hours with strict body temperature monitoring. They were followed up in the outpatient clinic by a multidisciplinary team. Motor development assessments showed that both patients had normal motor development. These results support the use of neonatal hypothermia as a neuroprotective intervention in infants with perinatal asphyxia, minimizing and preventing sequelae in motor development.
	
	\newpage
	
	% Bibliografia automática
	\printbibliography[title=References]
		
\end{document}