\documentclass[11pt,a4paper]{article}

% Apenas pacotes essenciais
\usepackage[margin=2.5cm]{geometry}
\usepackage{setspace}

% Bibliografia com biblatex - versão simples
\usepackage[backend=biber,style=authoryear,sorting=nyt]{biblatex}
\addbibresource{reference.bib}

% Configurações básicas
\onehalfspacing
\setlength{\parindent}{1.5cm}

% Informações do artigo
\title{\textbf{Neonatal Hypothermia and Neonatal Anoxia}}

\author{
	\textbf{iTec/FURG}$^{1}$ \\
	\textbf{Calvimontes S. C.}$^{2}$ \\
	\textbf{Copilot}$^{3}$ \\
	\\
	\small $^{1}$Centro de Robótica e Ciência de Dados \\
	\small Universidade Federal do Rio Grande (unidade EMbrapii / FURG) \\
	\small Rio Grande do Sul (Brazil) \\
	\\
	\small $^{2}$Universidade Federal Fluminense \\
	\small Email: sergiocc@id.uff.br \\
	\\
	\small $^{3}$Microsoft - Redmond, Washington (EUA) \\
	\small Web site: https://copilot.microsoft.com//
}

\date{\today}

\begin{document}
	
	\maketitle
	
	\newpage
	
	\section{Introduction}
	
	Therapeutic hypothermia is a neuroprotective strategy that reduces mortality and disability in newborns with hypoxic-ischemic encephalopathy resulting from perinatal asphyxia. The therapy should begin within the first six hours after birth and involves reducing the neonate’s body temperature to an average of 33–34°C for 72 hours [\cite{Azzopardi2014,Thayyil2021,Abate2021}]. Hypothermia decreases brain metabolism by approximately 5\% for every 1°C reduction in body temperature, delaying the onset of cellular anoxic depolarization [\cite{Silveira2015}].
		
	\subsection{Research Objectives}
	This study aims to report two clinical cases describing the effects of neonatal hypothermia in infants with perinatal asphyxia and their motor development during a follow-up program after hospital discharge.
	
	\section{Methods}
	This retrospective case report involves two children diagnosed with hypoxic-ischemic encephalopathy due to neonatal asphyxia who underwent a hypothermia protocol in the Neonatal Intensive Care Unit (NICU). Data from the prenatal, perinatal, and postnatal periods were collected from medical records. Subsequently, guardians were interviewed using a semi-structured maternal history guide, which included general information about the mother and child. The children were followed in a high-risk outpatient clinic and evaluated using the Hammersmith Infant Neurological Examination (HINE), the Alberta Infant Motor Scale (AIMS), and the Denver II screening test. All instruments were administered according to their respective manuals by trained evaluators. The study was approved by the University's Research Ethics Committee.
	
	\subsection{Case Description}
	\subsubsection{Case 1}
	The first case involved a female newborn delivered via cesarean section at 37 weeks of gestation, weighing 3,055 g and measuring 46.5 cm in length. Apgar scores were 5 and 6 at one and five minutes, respectively, requiring a cycle of positive pressure ventilation. The infant developed respiratory distress and received 20\% oxygen for one hour, followed by three hours of CPAP. At four hours of life, respiratory distress worsened, and cyanosis appeared in the extremities. The patient was intubated and exhibited hyperextension of the upper limbs, internal wrist rotation, and seizure activity during intubation. Based on diagnostic tests indicating perinatal asphyxia, therapeutic hypothermia was initiated by turning off the crib heater until the target temperature of 32–35°C was reached. Temperature was monitored every 20 minutes, and hypothermia was maintained for 74 hours. The infant was diagnosed with late neonatal sepsis and received six days of antibiotics. Transfontanellar ultrasound revealed reduced sulci and diffuse hyperechogenicity. After seven days, cranial magnetic resonance imaging showed sequelae of a severe hypoxic-ischemic event. The patient remained in the NICU for 12 days and in the ward for 10 days, and was discharged on a mixed diet of breast milk and formula. Neurological examination at discharge showed mild generalized hypotonia and symmetrical primitive reflexes (search, palmar and plantar grasp, complete Moro, and tonic neck reflexes). At the current age of 3 years and 3 months, motor development assessments conducted by the physiotherapy team indicate normal development for age.
		
	\section{Conclusion}
	These cases involved two children diagnosed with hypoxic-ischemic encephalopathy due to perinatal asphyxia who received therapeutic hypothermia for 74 hours with strict temperature monitoring. Follow-up assessments in the outpatient clinic revealed normal motor development in both patients. These findings support the use of neonatal hypothermia as a neuroprotective intervention to minimize and prevent motor development sequelae in infants with perinatal asphyxia.
	
	\newpage
	
	% Bibliografia automática
	\printbibliography[title=References]
		
\end{document}