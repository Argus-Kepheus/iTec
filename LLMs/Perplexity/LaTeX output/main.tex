\documentclass[11pt,a4paper]{article}

% Apenas pacotes essenciais
\usepackage[margin=2.5cm]{geometry}
\usepackage{setspace}

% Bibliografia com biblatex - versão simples
\usepackage[backend=biber,style=authoryear,sorting=nyt]{biblatex}
\addbibresource{reference.bib}

% Configurações básicas
\onehalfspacing
\setlength{\parindent}{1.5cm}

% Informações do artigo
\title{\textbf{NEONATAL HYPOTHERMIA AND NEONATAL ANOXIA}}

\author{
	\textbf{iTec/FURG}$^{1}$ \\
	\textbf{Calvimontes S. C.}$^{2}$ \\
	\textbf{Sonar}$^{3}$ \\
	\\
	\small $^{1}$Centro de Robótica e Ciência de Dados \\
	\small Universidade Federal do Rio Grande (unidade EMbrapii / FURG) \\
	\small Rio Grande do Sul (Brazil) \\
	\\
	\small $^{2}$Universidade Federal Fluminense \\
	\small Email: sergiocc@id.uff.br \\
	\\
	\small $^{3}$Perplexity - San Francisco, Califórnia (EUA) \\
	\small Web site: https://www.perplexity.ai/
}

\date{\today}

\begin{document}
	
	\maketitle
	
	\newpage
	
	\section{Introduction}
	Therapeutic hypothermia is a neuroprotective strategy that reduces mortality and disability in newborns with hypoxic-ischemic encephalopathy due to perinatal asphyxia. The therapy should begin within the first six hours after birth and consists of reducing the body temperature of neonates to an average of 33ºC–34ºC for 72 hours [\cite{Azzopardi2014,Thayyil2021,Abate2021}]. Hypothermia reduces brain metabolism by approximately 5\% for every 1°C decrease in body temperature, delaying the onset of cellular anoxic depolarization [\cite{Silveira2015}].
		
	\subsection{Research Objectives}
	This study reports two clinical cases to describe the effects of neonatal hypothermia in infants with perinatal asphyxia and motor development in a follow-up program after hospital discharge.
	
	\section{Methods}
	This retrospective case report involves two children diagnosed with hypoxic-ischemic encephalopathy due to neonatal asphyxia who were submitted to a hypothermia protocol in the Neonatal Intensive Care Unit (NICU). Data on the prenatal, perinatal, and postnatal periods were collected from the children’s medical records. Subsequently, an interview was conducted with the guardian using a semi-structured maternal history guide, including general information about the mother and baby. The children were followed up at the high-risk outpatient clinic and evaluated using the Hammersmith Neurological Examination (HINE), motor development assessment using the Alberta Infant Motor Scale (AIMS), and the Denver II screening test. The instruments were administered according to the recommendations in the assessment manuals by trained evaluators. The study was approved by the University’s Research Ethics Committee.
	
	\subsection{Case Description}
	\subsubsection{Case 1}
	A female newborn was born by cesarean section at 37 weeks of gestational age, weighing 3,055 g and measuring 46.5 cm. The patient had Apgar scores of 5 and 6 at the first and fifth minutes, respectively, requiring a cycle of positive pressure ventilation (PPV). The infant developed respiratory distress and was given 20\% oxygen for one hour, followed by three hours of continuous positive airway pressure (CPAP). After four hours of life, the patient worsened respiratory distress and developed cyanosis in the extremities, requiring intubation. During intubation, she experienced an episode of upper limb hyperextension, wrist internal rotation, and seizure.
	
	Due to diagnostic tests indicating perinatal asphyxia, the therapeutic hypothermia protocol was initiated by turning off the incubator until the patient’s temperature reached the ideal range of 32º–35ºC, monitored every 20 minutes, and maintained for 74 hours. The baby was diagnosed with late neonatal sepsis in the NICU and required six days of antibiotics. Transfontanellar ultrasound revealed sulcal reduction and diffuse hyperechogenicity. After seven days, cranial magnetic resonance imaging (MRI) demonstrated sequelae of a severe perinatal hypoxic-ischemic event. The patient remained 12 days in the NICU and 10 days in the ward, being discharged on a combined breast milk and formula diet. At discharge, neurological examination showed mild generalized hypotonia and symmetrical primitive reflexes (search reflex, palm and plantar handgrip, complete Moro reflex, and tonic-cervical reflex). Currently, at 3 years and 3 months of age, evaluations by the physiotherapy team at the pediatric outpatient clinic indicate motor development within the normal range for age.
		
	\section{Conclusion}
	The two cases involved children diagnosed with hypoxic-ischemic encephalopathy due to perinatal asphyxia who underwent a 74-hour therapeutic hypothermia protocol with strict body temperature monitoring. They were followed by a multidisciplinary outpatient team, and motor development assessment showed normal outcomes in both. These results support the use of neonatal hypothermia as a neuroprotective intervention in infants with perinatal asphyxia, minimizing and preventing motor development sequelae.
	
	\newpage
	
	% Bibliografia automática
	\printbibliography[title=References]
		
\end{document}