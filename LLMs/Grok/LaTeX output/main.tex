\documentclass[11pt,a4paper]{article}

% Apenas pacotes essenciais
\usepackage[margin=2.5cm]{geometry}
\usepackage{setspace}

% Bibliografia com biblatex - versão simples
\usepackage[backend=biber,style=authoryear,sorting=nyt]{biblatex}
\addbibresource{reference.bib}

% Configurações básicas
\onehalfspacing
\setlength{\parindent}{1.5cm}

% Informações do artigo
\title{\textbf{Neonatal Hypothermia and Neonatal Anoxia}}

\author{
	\textbf{iTec/FURG}$^{1}$ \\
	\textbf{Calvimontes S. C.}$^{2}$ \\
	\textbf{Grok}$^{3}$ \\
	\\
	\small $^{1}$Centro de Robótica e Ciência de Dados \\
	\small Universidade Federal do Rio Grande (unidade EMbrapii / FURG) \\
	\small Rio Grande do Sul (Brazil) \\
	\\
	\small $^{2}$Universidade Federal Fluminense \\
	\small Email: sergiocc@id.uff.br \\
	\\
	\small $^{3}$xAI - Palo Alto, Califórnia, (EUA) \\
	\small Web site: https://grok.com/
}

\date{\today}

\begin{document}
	
	\maketitle
	
	\newpage
	
	\section{Introduction}
	Therapeutic hypothermia is a neuroprotective strategy that reduces mortality and disability in newborns with hypoxic-ischemic encephalopathy due to perinatal asphyxia. The therapy should start within the first 6 hours after birth and consists of reducing the neonatal core body temperature (to 33--34°C) for 72 hours [\cite{Azzopardi2014,Thayyil2021,Abate2021}]. Hypothermia reduces brain metabolism by approximately 5\% for every 1°C decrease in body temperature, which delays the onset of cellular anoxic depolarization [\cite{Silveira2015}].
		
	\subsection{Research Objectives}
	The goal of this study was to report two clinical cases describing the effects of therapeutic neonatal hypothermia in infants with perinatal asphyxia and their motor development during follow-up after hospital discharge.
	
	\section{Methods}
	This retrospective case report involved two children diagnosed with hypoxic-ischemic encephalopathy due to neonatal asphyxia who underwent a hypothermia protocol in the Neonatal Intensive Care Unit (NICU). Data on the prenatal, perinatal, and postnatal periods were collected from the medical records. Subsequently, a semi-structured interview was conducted with the guardian using a maternal history guide that included general information about the mother and infant. The children were followed in the high-risk outpatient clinic and evaluated using the Hammersmith Neurological Examination (HNE), the Alberta Infant Motor Scale (AIMS) for motor development assessment, and the Denver II screening test. The instruments were administered according to the recommendations in the assessment manuals by trained evaluators. The study was approved by the University's Research Ethics Committee.
	
	\subsection{Case Description}
	\subsubsection{Case 1}
	A female newborn was delivered by cesarean section at 37 weeks of gestation, with a birth weight of 3055 g and length of 46.5 cm. The Apgar scores were 5 and 6 at the first and fifth minutes, respectively, requiring one cycle of positive pressure ventilation (PPV). The infant developed respiratory distress; 20\% oxygen was delivered for 1 hour, followed by 3 hours of continuous positive airway pressure (CPAP). At 4 hours of life, respiratory distress worsened with cyanosis in the extremities; the infant was intubated and, during the procedure, presented an episode of upper limb hyperextension, internal wrist rotation, and seizure. Due to evidence of perinatal asphyxia, the therapeutic hypothermia protocol was initiated by turning off the incubator until the target temperature of 33--35°C was reached, with monitoring every 20 min, and maintained for 74 hours. The infant was diagnosed with late-onset neonatal sepsis in the NICU and received antibiotics for 6 days. Transfontanellar ultrasound showed sulcal narrowing and diffuse hyperechogenicity. After 7 days, cranial magnetic resonance imaging (MRI) demonstrated sequelae of a severe perinatal hypoxic-ischemic event. The infant remained 12 days in the NICU and 10 days in the ward before discharge on breast milk and formula. At discharge, neurological examination revealed mild generalized hypotonia with symmetrical primitive reflexes (search, palmar grasp, plantar grasp, complete Moro, and tonic neck reflexes present). Currently, the child is 3 years and 3 months old, and evaluations by the physiotherapy team in the pediatric outpatient clinic demonstrate motor development within the normal range for age.
		
	\section{Conclusion}
	The two cases involved children diagnosed with hypoxic-ischemic encephalopathy due to perinatal asphyxia who received therapeutic hypothermia for 74 hours with strict body temperature monitoring. They were followed by a multidisciplinary team in the outpatient clinic, and motor development assessments showed normal development in both patients. These results support the use of neonatal hypothermia as a neuroprotective intervention in infants with perinatal asphyxia to minimize and prevent motor sequelae.
	
	\newpage
	
	% Bibliografia automática
	\printbibliography[title=References]
		
\end{document}