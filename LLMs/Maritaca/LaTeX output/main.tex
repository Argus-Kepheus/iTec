\documentclass[11pt,a4paper]{article}

% Apenas pacotes essenciais
\usepackage[margin=2.5cm]{geometry}
\usepackage{setspace}

% Bibliografia com biblatex - versão simples
\usepackage[backend=biber,style=authoryear,sorting=nyt]{biblatex}
\addbibresource{reference.bib}

% Configurações básicas
\onehalfspacing
\setlength{\parindent}{1.5cm}

% Informações do artigo
\title{\textbf{Therapeutic Hypothermia and Neonatal Anoxia: Case Reports on Neuroprotection and Motor Development}}

\author{
	\textbf{iTec/FURG}$^{1}$ \\
	\textbf{Calvimontes S. C.}$^{2}$ \\
	\textbf{Sabiá-3.1}$^{3}$ \\
	\\
	\small $^{1}$Centro de Robótica e Ciência de Dados \\
	\small Universidade Federal do Rio Grande (unidade EMbrapii / FURG) \\
	\small Rio Grande do Sul (Brazil) \\
	\\
	\small $^{2}$Universidade Federal Fluminense \\
	\small Email: sergiocc@id.uff.br \\
	\\
	\small $^{3}$Maritaca AI - Campinas, São Paulo (Brazil) \\
	\small Web site: https://chat.maritaca.ai/
}

\date{\today}

\begin{document}
	
	\maketitle
	
	\newpage
	
	\section{Introduction}
	Therapeutic hypothermia is a neuroprotective strategy that reduces mortality and disability in newborns with hypoxic-ischemic encephalopathy (HIE) resulting from perinatal asphyxia. The therapy should commence within the first six hours after birth and involves reducing the body temperature of neonates to an average of 33–34°C for 72 hours [\cite{Azzopardi2014, Laptook2017, Silveira2015}]. Hypothermia decreases brain metabolism by approximately 5\% for every 1°C reduction in body temperature, thereby delaying the onset of cellular anoxic depolarization [\cite{Silveira2015}].
		
	\subsection{Research Objectives}
	The aim of this study is to report two clinical cases describing the effects of neonatal hypothermia on babies with perinatal asphyxia and their subsequent motor development in a follow-up program after hospital discharge.
	
	\section{Methods}
	A female newborn, delivered by cesarean section at 37 weeks of gestational age, weighing 3055 g and measuring 46.5 cm, presented an Apgar score of 5 and 6 at the first and fifth minutes, respectively, requiring positive pressure ventilation (PPV). The infant developed respiratory distress and received 20\% oxygen for 1 hour, followed by 3 hours of continuous positive airway pressure (CPAP). Four hours after birth, the patient's respiratory distress worsened, with cyanosis in the extremities, leading to intubation. During intubation, she exhibited an episode of hyperextension of the upper limbs, internal rotation of the wrists, and seizure. Given the diagnosis of perinatal asphyxia, the therapeutic hypothermia protocol was initiated by turning off the crib heating until the patient reached an ideal temperature of 32–35°C, monitored every 20 minutes, and maintained for 74 hours. A transfontanellar ultrasound indicated reduced sulci and diffuse hyperechogenicity. After seven days, a cranial magnetic resonance imaging (MRI) demonstrated sequelae from a severe hypoxic-ischemic event. The patient stayed 12 days in the NICU and 10 days in the ward, being discharged with a diet consisting of breast milk and formula. At discharge, the neurological examination revealed mild generalized hypotonia and symmetrical primitive reflexes (rooting, palmar and plantar grasp, and complete Moro and tonic-cervical reflexes). Currently, the child is 3 years and 3 months old, and assessments by the physiotherapy team at the pediatric outpatient clinic indicate motor development within the normal range for her age.
	
	\subsection{Case Description}
	\subsubsection{Case 1}
	The first case involved a female newborn delivered by cesarean section at 37 weeks of gestation, weighing 3,055 g and measuring 46.5 cm in length. The patient presented Apgar scores of 5 and 6 at the first and fifth minutes, respectively, requiring a cycle of positive pressure ventilation (PPV). The infant developed respiratory distress and received 20\% oxygen for one hour, followed by three hours of continuous positive airway pressure (CPAP).
	After four hours of life, the patient’s respiratory distress worsened, and cyanosis appeared in the extremities. She was intubated, during which she presented an episode of upper-limb hyperextension, wrist internal rotation, and seizure activity. Based on the tests confirming perinatal asphyxia, the therapeutic hypothermia protocol was initiated by turning off the crib heater until the patient’s body temperature reached the target range of 32–35°C. Temperature was monitored every 20 minutes, and the cooling was maintained for 74 hours.
	During hospitalization, the patient developed late neonatal sepsis in the NICU and required six days of antibiotic therapy. Cranial transfontanellar ultrasound revealed sulcal narrowing and diffuse hyperechogenicity. After seven days, cranial magnetic resonance imaging (MRI) demonstrated sequelae of a severe perinatal hypoxic-ischemic event. The infant remained in the NICU for 12 days and in the ward for 10 days, being discharged with a combined breast and formula diet.
	At discharge, neurological examination showed mild generalized hypotonia, with symmetric primitive reflexes (rooting, palmar and plantar grasp, Moro, and tonic-neck reflexes). Currently, the child is three years and three months old, and follow-up evaluations by the physiotherapy team indicate normal motor development for her age.
		
	\section{Conclusion}
	The cases presented involved two children diagnosed with hypoxic-ischemic encephalopathy due to perinatal asphyxia who received a therapeutic hypothermia protocol for 74 hours with strict monitoring of body temperature. Follow-up assessments by the multidisciplinary team showed that both patients had normal motor development. These results support the use of the neonatal hypothermia protocol as a neuroprotective intervention in babies with perinatal asphyxia, potentially minimizing and preventing sequelae in children's motor development.
	
	\newpage
	
	% Bibliografia automática
	\printbibliography[title=References]
		
\end{document}