\documentclass[11pt,a4paper]{article}

% Apenas pacotes essenciais
\usepackage[margin=2.5cm]{geometry}
\usepackage{setspace}

% Bibliografia com biblatex - versão simples
\usepackage[backend=biber,style=authoryear,sorting=nyt]{biblatex}
\addbibresource{reference.bib}

% Configurações básicas
\onehalfspacing
\setlength{\parindent}{1.5cm}

% Informações do artigo
\title{\textbf{Neonatal Hypothermia and Neonatal Anoxia: Clinical Outcomes and Motor Development Following Therapeutic Hypothermia}}

\author{
	\textbf{iTec/FURG}$^{1}$ \\
	\textbf{Calvimontes S. C.}$^{2}$ \\
	\textbf{Le Chat}$^{3}$ \\
	\\
	\small $^{1}$Centro de Robótica e Ciência de Dados \\
	\small Universidade Federal do Rio Grande (unidade EMbrapii / FURG) \\
	\small Rio Grande do Sul (Brazil) \\
	\\
	\small $^{2}$Universidade Federal Fluminense \\
	\small Email: sergiocc@id.uff.br \\
	\\
	\small $^{3}$Mistral AI - Paris (France) \\
	\small Web site: https://chat.mistral.ai/
}

\date{\today}

\begin{document}
	
	\maketitle
	
	\newpage
	
	\section{Introduction}
	Therapeutic hypothermia is a neuroprotective strategy that reduces mortality and disability in newborns with hypoxic-ischemic encephalopathy resulting from perinatal asphyxia. Therapy should commence within the first six hours after birth and involves reducing the neonate’s body temperature to an average of 33–34°C for 72 hours [\cite{Azzopardi2014,Thayyil2021,Abate2021}]. Hypothermia reduces brain metabolism by approximately 5\% for every 1°C decrease in body temperature, thereby delaying the onset of cellular anoxic depolarization [\cite{Silveira2015}].
		
	\subsection{Research Objectives}
	This study reports two clinical cases describing the effects of neonatal hypothermia in infants with perinatal asphyxia and their motor development outcomes in a follow-up program after hospital discharge.
	
	\section{Methods}
	This retrospective case report involved two children diagnosed with hypoxic-ischemic encephalopathy due to neonatal asphyxia who underwent a hypothermia protocol in the Neonatal Intensive Care Unit (NICU). Data regarding the prenatal, perinatal, and postnatal periods were collected from the children’s medical records. A semi-structured interview with the guardian was conducted to gather maternal and infant history. The children were followed up in a high-risk outpatient clinic and evaluated using the Hammersmith Neurological Examination (HINE), the Alberta Infant Motor Scale (AIMS), and the Denver II screening test. All assessments were administered by trained evaluators according to standardized protocols. The study was approved by the University’s Research Ethics Committee.
	
	\subsection{Case Description}
	\subsubsection{Case 1}
	A female newborn, delivered via cesarean section at 37 weeks’ gestational age, weighed 3,055 g and measured 46.5 cm in length. The infant’s Apgar scores were 5 and 6 at the first and fifth minutes, respectively, necessitating positive pressure ventilation (PPV). She developed respiratory distress and received 20\% oxygen for 1 hour, followed by 3 hours of continuous positive airway pressure (CPAP). At 4 hours of life, her condition worsened, with cyanosis in the extremities, leading to intubation. During intubation, she exhibited hyperextension of the upper limbs, internal rotation of the wrists, and a seizure episode.
	Due to confirmed perinatal asphyxia, the therapeutic hypothermia protocol was initiated. The incubator was adjusted until the infant’s temperature reached 33–34°C, monitored every 20 minutes, and maintained for 74 hours. The infant was diagnosed with late-onset neonatal sepsis and required 6 days of antibiotic therapy. A transfontanellar ultrasound revealed reduced sulci and diffuse hyperechogenicity. A cranial magnetic resonance imaging (MRI) scan performed after seven days confirmed sequelae of a severe hypoxic-ischemic event.
	The infant remained in the NICU for 12 days and in the ward for 10 days before discharge on a diet of breast milk and formula. At discharge, neurological examination revealed mild generalized hypotonia and symmetrical primitive reflexes (search, palmar/plantar grasp, Moro, and tonic neck reflexes). Currently, at 3 years and 3 months of age, follow-up evaluations by the physiotherapy team indicate motor development within the normal range for her age.
		
	\section{Conclusion}
	The presented cases involved two infants diagnosed with hypoxic-ischemic encephalopathy due to perinatal asphyxia who underwent a 74-hour therapeutic hypothermia protocol with strict temperature monitoring. Follow-up assessments by a multidisciplinary team revealed normal motor development in both patients. These results support the use of neonatal hypothermia as a neuroprotective intervention to minimize and prevent motor sequelae in infants with perinatal asphyxia.
	
	\newpage
	
	% Bibliografia automática
	\printbibliography[title=References]
		
\end{document}