\documentclass[11pt,a4paper]{article}

% Apenas pacotes essenciais
\usepackage[margin=2.5cm]{geometry}
\usepackage{setspace}

% Bibliografia com biblatex - versão simples
\usepackage[backend=biber,style=authoryear,sorting=nyt]{biblatex}
\addbibresource{reference.bib}

% Configurações básicas
\onehalfspacing
\setlength{\parindent}{1.5cm}

% Informações do artigo
\title{\textbf{Neonatal Hypothermia and Neonatal Anoxia}}

\author{
	\textbf{iTec/FURG}$^{1}$ \\
	\textbf{Calvimontes S. C.}$^{2}$ \\
	\textbf{Qwen}$^{3}$ \\
	\\
	\small $^{1}$Centro de Robótica e Ciência de Dados \\
	\small Universidade Federal do Rio Grande (unidade EMbrapii / FURG) \\
	\small Rio Grande do Sul (Brazil) \\
	\\
	\small $^{2}$Universidade Federal Fluminense \\
	\small Email: sergiocc@id.uff.br \\
	\\
	\small $^{3}$Alibaba Cloud - Hangzhou, Zhejiang (China) \\
	\small Web site: https://chat.qwen.ai/
}

\date{\today}

\begin{document}
	
	\maketitle
	
	\newpage
	
	\section{Introduction}
	Therapeutic hypothermia is a neuroprotective strategy that reduces mortality and disability in newborns with hypoxic–ischemic encephalopathy resulting from perinatal asphyxia. The therapy should start within the first six hours after birth and consists of reducing the body temperature of neonates (average of 33–34°C) for 72 hours [\cite{Azzopardi2014,Laptook2017,Thayyil2021}]. Hypothermia reduces brain metabolism by approximately 5\% for every 1°C decrease in body temperature, which delays the onset of cellular anoxic depolarization [\cite{Silveira2015}].
		
	\subsection{Research Objectives}
	The goal of this study was to report two clinical cases describing the effects of neonatal hypothermia in infants with perinatal asphyxia and their motor development during a follow-up program after hospital discharge.
	
	\section{Methods}
	This is a retrospective case report involving two children diagnosed with hypoxic–ischemic encephalopathy due to neonatal asphyxia and submitted to a hypothermia protocol in the Neonatal Intensive Care Unit (NICU). Data regarding the prenatal, perinatal, and postnatal periods were collected from the children's medical records. Subsequently, an interview with the guardian was conducted using a semi-structured maternal history guide, including general information about the mother and infant. The children were followed up in the high-risk outpatient clinic and evaluated using the Hammersmith Infant Neurological Examination (HINE), motor development assessment using the Alberta Infant Motor Scale (AIMS), and the Denver II screening test. The instruments were administered according to the recommendations in the assessment manuals and were administered by trained evaluators. The study was approved by the University's Research Ethics Committee.
	
	\subsection{Case Description}
	\subsubsection{Case 1}
	A female newborn was delivered by cesarean section at 37 weeks of gestational age, weighing 3,055 g and measuring 46.5 cm in length. The patient presented Apgar scores of 5 and 6 at 1 and 5 minutes, respectively, requiring one cycle of positive pressure ventilation (PPV). The infant developed respiratory distress; thus, 20\% oxygen was administered for 1 hour, followed by 3 hours of continuous positive airway pressure (CPAP). After 4 hours of life, the patient exhibited worsening respiratory distress and cyanosis in the extremities, necessitating intubation. During intubation, she experienced an episode of upper limb hyperextension, wrist internal rotation, and seizure.
	
	Due to clinical and laboratory findings consistent with perinatal asphyxia, the therapeutic hypothermia protocol was initiated by turning off the incubator until the patient reached the target temperature range of 32–35°C, with monitoring every 20 minutes, and maintained for 74 hours. The infant was diagnosed with late-onset neonatal sepsis in the NICU and required 6 days of antibiotic therapy. Transfontanelle ultrasound revealed reduced sulcation and diffuse hyperechogenicity. After 7 days, cranial magnetic resonance imaging (MRI) demonstrated sequelae of a severe perinatal hypoxic–ischemic event.
	
	The patient remained in the NICU for 12 days and in the ward for 10 days, and was discharged receiving both breast milk and formula. At discharge, the neurological examination showed mild generalized hypotonia with present and symmetrical primitive reflexes (rooting reflex, palmar and plantar grasp, and complete Moro and asymmetric tonic neck reflexes). Currently, the child is 3 years and 3 months old, and evaluations conducted by the physiotherapy team at the pediatric outpatient clinic indicate motor development within the normal range for age.
		
	\section{Conclusion}
	The cases presented involved two children diagnosed with hypoxic–ischemic encephalopathy due to perinatal asphyxia who received a therapeutic hypothermia protocol for 74 hours with strict monitoring of body temperature. They were followed up in the outpatient clinic by a multidisciplinary team, and motor development assessments indicated that both patients exhibited normal motor development. These favorable outcomes support the use of neonatal hypothermia as a neuroprotective intervention in infants with perinatal asphyxia, minimizing and preventing sequelae in motor development.
	
	\newpage
	
	% Bibliografia automática
	\printbibliography[title=References]
		
\end{document}