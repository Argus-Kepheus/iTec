\documentclass[11pt,a4paper]{article}

% Apenas pacotes essenciais
\usepackage[margin=2.5cm]{geometry}
\usepackage{setspace}

% Bibliografia com biblatex - versão simples
\usepackage[backend=biber,style=authoryear,sorting=nyt]{biblatex}
\addbibresource{reference.bib}

% Configurações básicas
\onehalfspacing
\setlength{\parindent}{1.5cm}

% Informações do artigo
\title{\textbf{Neonatal Hypothermia and Neonatal Anoxia}}

\author{
	\textbf{iTec/FURG}$^{1}$ \\
	\textbf{Calvimontes S. C.}$^{2}$ \\
	\textbf{Claude Sonnet 4.5}$^{3}$ \\
	\\
	\small $^{1}$Centro de Robótica e Ciência de Dados \\
	\small Universidade Federal do Rio Grande (unidade EMbrapii / FURG) \\
	\small Rio Grande do Sul (Brazil) \\
	\\
	\small $^{2}$Universidade Federal Fluminense \\
	\small Email: sergiocc@id.uff.br \\
	\\
	\small $^{3}$Anthropic - San Francisco, Califórnia (EUA) \\
	\small Web site: https://claude.ai//
}

\date{\today}

\begin{document}
	
	\maketitle
	
	\newpage
	
	\section{Introduction}
	
	Therapeutic hypothermia is a neuroprotective strategy that reduces mortality and disability in newborns with hypoxic-ischemic encephalopathy from perinatal asphyxia. The therapy should start within the first six hours after birth and consists of reducing the body temperature of neonates to an average of 33–34°C for 72 hours [\cite{Azzopardi2014,Thayyil2021,Abate2021}]. Hypothermia reduces brain metabolism by approximately 5\% for every 1°C decrease in body temperature, which delays the onset of cellular anoxic depolarization [\cite{Silveira2015}].
		
	\subsection{Research Objectives}
	The goal of this study was to report two clinical cases describing the effects of neonatal hypothermia in babies with perinatal asphyxia and motor development in a follow-up program after hospital discharge.
	
	\section{Methods}
	This is a retrospective case report involving two children diagnosed with hypoxic-ischemic encephalopathy due to neonatal asphyxia and submitted to a hypothermia protocol in the Neonatal Intensive Care Unit (NICU). Data regarding the prenatal, perinatal, and postnatal periods were collected from the children's medical records. Subsequently, an interview with the guardian was conducted using a semi-structured maternal history guide, including general information about the mother and baby. The children were followed up in the high-risk outpatient clinic and evaluated using the Hammersmith Neurological Examination (HINE), motor development assessment using the Alberta Infant Motor Scale (AIMS), and the Denver II screening test. The instruments were administered according to the recommendations in the assessment manuals and were applied by trained evaluators. The study was approved by the University's Research Ethics Committee.
	
	\subsection{Case Description}
	\subsubsection{Case 1}
	A female newborn was born by cesarean section at 37 weeks of gestational age, weighing 3055 g and measuring 46.5 cm in length. The patient presented an Apgar score of 5 and 6 at the first and fifth minutes, respectively, requiring a cycle of positive pressure ventilation. The infant evolved with respiratory distress; thus, 20\% oxygen was delivered for one hour, followed by three hours of continuous positive airway pressure. After four hours of life, the patient presented worsening of respiratory distress and the presence of cyanosis in the extremities, being intubated. During intubation, she presented an episode of hyperextension of the upper limbs, internal rotation of the wrists, and seizure. Due to the tests that showed perinatal asphyxia, the therapeutic hypothermia protocol was started by turning off the warmer until the patient reached the target temperature of 33–34°C, being monitored every 20 minutes and remaining at this temperature for 74 hours. The baby was diagnosed with late neonatal sepsis in the NICU and required six days of antibiotics. Transfontanellar ultrasound was performed, indicating a reduction of the sulci and diffuse hyperechogenicity. After seven days, cranial magnetic resonance imaging demonstrated sequelae of severe perinatal hypoxic-ischemic injury. The patient remained 12 days in the NICU and 10 days in the ward, being discharged with a diet of breast milk and formula. In the neurological examination at discharge, the patient presented mild generalized hypotonia and primitive reflexes that were present and symmetrical (rooting reflex, palmar and plantar grasp, and complete Moro and tonic neck reflexes present). Currently, the child has a chronological age of three years and three months, and evaluations conducted by the physiotherapy team at the pediatric outpatient clinic demonstrated motor development within the normal range for age.
		
	\section{Conclusion}
	The cases presented involved two children diagnosed with hypoxic-ischemic encephalopathy due to perinatal asphyxia who received a therapeutic protocol of hypothermia for 74 hours with strict monitoring of body temperature. They were followed up at the outpatient clinic by the multidisciplinary team, and in the assessment of motor development, it was observed that both patients had normal motor development. The results obtained are favorable for the use of the neonatal hypothermia protocol as a neuroprotective intervention in babies with perinatal asphyxia, minimizing and preventing sequelae in children's motor development.
	
	\newpage
	
	% Bibliografia automática
	\printbibliography[title=References]
		
\end{document}