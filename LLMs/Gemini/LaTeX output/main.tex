\documentclass[11pt,a4paper]{article}

% Apenas pacotes essenciais
\usepackage[margin=2.5cm]{geometry}
\usepackage{setspace}

% Bibliografia com biblatex - versão simples
\usepackage[backend=biber,style=authoryear,sorting=nyt]{biblatex}
\addbibresource{reference.bib}

% Configurações básicas
\onehalfspacing
\setlength{\parindent}{1.5cm}

% Informações do artigo
\title{\textbf{Neonatal Hypothermia and Hypoxic-Ischemic Encephalopathy}}

\author{
	\textbf{iTec/FURG}$^{1}$ \\
	\textbf{Calvimontes S. C.}$^{2}$ \\
	\textbf{Gemini}$^{3}$ \\
	\\
	\small $^{1}$Centro de Robótica e Ciência de Dados \\
	\small Universidade Federal do Rio Grande (unidade EMbrapii / FURG) \\
	\small Rio Grande do Sul (Brazil) \\
	\\
	\small $^{2}$Universidade Federal Fluminense \\
	\small Email: sergiocc@id.uff.br \\
	\\
	\small $^{3}$Google - Mountain View, Califórnia (EUA) \\
	\small Web site: https://gemini.google.com/
}

\date{\today}

\begin{document}
	
	\maketitle
	
	\newpage
	
	\section{Introduction}
	Therapeutic hypothermia is a neuroprotective strategy that reduces mortality and disability of newborns with hypoxic-ischemic encephalopathy from perinatal asphyxia. The therapy should start within the first 6 hours after birth and consists of reducing the neonate's body temperature (average of $33-34^{\circ}$C) for $72$ hours [\cite{Azzopardi2014, Thayyil2021, Abate2021}]. Hypothermia reduces brain metabolism by approximately $5\%$ for every $1^{\circ}$C decrease in body temperature, which delays the onset of cellular anoxic depolarization [\cite{Silveira2015}].
		
	\subsection{Research Objectives}
	The goal of this study was to report two clinical cases describing the effects of neonatal hypothermia in babies with perinatal asphyxia and their motor development in a follow-up program after hospital discharge.
	
	\section{Methods}
	This was a retrospective case report involving two children diagnosed with hypoxic-ischemic encephalopathy due to neonatal asphyxia and submitted to a hypothermia protocol in the Neonatal Intensive Care Unit (NICU). Data regarding the prenatal, perinatal, and postnatal periods were collected from the children's medical records. Subsequently, an interview with the guardian was conducted using a semi-structured maternal history guide, which included general information about the mother and baby. The children were followed up in the high-risk outpatient clinic and evaluated using the Hammersmith Neurological Examination (HINE), the Alberta Infant Motor Scale (AIMS), and the Denver II screening test. The instruments were administered by trained evaluators according to the recommendations in the assessment manuals. The study was approved by the University Research Ethics Committee.
	
	\subsection{Case Description}
	\subsubsection{Case 1}
	A female newborn, born by cesarean section at 37 weeks of gestational age, weighed $3,055$ g and had a length of $46.5$ cm. The patient presented Apgar scores of 5 and 6 at 1 and 5 minutes, respectively, and required a cycle of positive pressure ventilation (PPV). The infant evolved with respiratory distress; thus, $20\%$ oxygen was delivered for $1$ hour, followed by $3$ hours of CPAP. After $4$ hours of life, the patient presented worsening respiratory distress and cyanosis in the extremities. She was intubated, and during the procedure, she experienced an episode of hyperextension of the upper limbs, internal rotation of the wrists, and a seizure. Given the tests indicating perinatal asphyxia, the therapeutic hypothermia protocol was initiated. This involved turning off the crib until the patient reached the ideal temperature of $32-35^{\circ}$C. The patient was monitored every $20$ min and remained in the protocol for $74$ hours. The baby was diagnosed with late neonatal sepsis in the NICU and required 6 days of antibiotics. Transfontanellar ultrasound was performed, indicating a reduction of the sulci and diffuse hyperechogenicity. After $7$ days, Cranial Magnetic Resonance (CMR) imaging demonstrated sequelae of a severe perinatal hypoxic-ischemic event. The patient remained $12$ days in the NICU and $10$ days in the ward. She was discharged with a diet of breast milk and formula. In the neurological examination at discharge, the patient presented mild generalized hypotonia and symmetrical primitive reflexes (search reflex, palm and plantar handgrip, and complete Moro and tonic-cervical reflexes were present). Currently, the child has a chronological age of 3 years and 3 months, and evaluations conducted by the physiotherapy team at the pediatric outpatient clinic demonstrate motor development within the normal range for the age.
		
	\section{Conclusion}
	The presented cases involved two children diagnosed with hypoxic-ischemic encephalopathy due to perinatal asphyxia. Both received a therapeutic protocol of hypothermia for $74$ hours with strict monitoring of body temperature. They were followed up at the outpatient clinic by a multidisciplinary team, and motor development assessment showed that both patients had normal motor development. The results obtained are favorable for the use of the neonatal hypothermia protocol as a neuroprotective intervention in babies with perinatal asphyxia, minimizing and preventing sequelae in children's motor development.
	
	\newpage
	
	% Bibliografia automática
	\printbibliography[title=References]
		
\end{document}