\documentclass[11pt,a4paper]{article}

% Apenas pacotes essenciais
\usepackage[margin=2.5cm]{geometry}
\usepackage{setspace}

% Bibliografia com biblatex - versão simples
\usepackage[backend=biber,style=authoryear,sorting=nyt]{biblatex}
\addbibresource{reference.bib}

% Configurações básicas
\onehalfspacing
\setlength{\parindent}{1.5cm}

% Informações do artigo
\title{\textbf{Neonatal Hypothermia and Neonatal Anoxia}}

\author{
	\textbf{iTec/FURG}$^{1}$ \\
	\textbf{Calvimontes S. C.}$^{2}$ \\
	\textbf{GPT-5}$^{3}$ \\
	\\
	\small $^{1}$Centro de Robótica e Ciência de Dados \\
	\small Universidade Federal do Rio Grande (unidade EMbrapii / FURG) \\
	\small Rio Grande do Sul (Brazil) \\
	\\
	\small $^{2}$Universidade Federal Fluminense \\
	\small Email: sergiocc@id.uff.br \\
	\\
	\small $^{3}$OpenAI - San Francisco, Califórnia (EUA) \\
	\small Web site: https://chatgpt.com/
}

\date{\today}

\begin{document}
	
	\maketitle
	
	\newpage
	
	\section{Introduction}
	
	Therapeutic hypothermia is a neuroprotective strategy that reduces mortality and disability in newborns with hypoxic-ischemic encephalopathy resulting from perinatal asphyxia. The therapy should begin within the first six hours after birth and consists of reducing the neonate’s body temperature to an average of 33–34°C for 72 hours [\cite{Azzopardi2014,Thayyil2021,Abate2021}]. Hypothermia decreases cerebral metabolism by approximately 5\% for each 1°C reduction in body temperature, thereby delaying the onset of cellular anoxic depolarization [\cite{Silveira2015}].
		
	\subsection{Research Objectives}
	This study reports two clinical cases describing the effects of neonatal hypothermia in infants with perinatal asphyxia and their motor development outcomes in a follow-up program after hospital discharge.
	
	\section{Methods}
	This is a retrospective case report involving two infants diagnosed with hypoxic-ischemic encephalopathy due to neonatal asphyxia who underwent a therapeutic hypothermia protocol in the Neonatal Intensive Care Unit (NICU). Data regarding the prenatal, perinatal, and postnatal periods were collected from the patients’ medical records. Subsequently, interviews were conducted with the guardians using a semi-structured maternal history questionnaire, including general information about the mother and infant.
	The children were followed in a high-risk outpatient clinic and evaluated using the Hammersmith Neurological Examination (HINE), the Alberta Infant Motor Scale (AIMS), and the Denver II screening test. All instruments were applied according to the respective manuals by trained evaluators. The study was approved by the University Research Ethics Committee.
	
	\subsection{Case Description}
	\subsubsection{Case 1}
	The first case involved a female newborn delivered by cesarean section at 37 weeks of gestation, weighing 3,055 g and measuring 46.5 cm in length. The patient presented Apgar scores of 5 and 6 at the first and fifth minutes, respectively, requiring a cycle of positive pressure ventilation (PPV). The infant developed respiratory distress and received 20\% oxygen for one hour, followed by three hours of continuous positive airway pressure (CPAP).
	After four hours of life, the patient’s respiratory distress worsened, and cyanosis appeared in the extremities. She was intubated, during which she presented an episode of upper-limb hyperextension, wrist internal rotation, and seizure activity. Based on the tests confirming perinatal asphyxia, the therapeutic hypothermia protocol was initiated by turning off the crib heater until the patient’s body temperature reached the target range of 32–35°C. Temperature was monitored every 20 minutes, and the cooling was maintained for 74 hours.
	During hospitalization, the patient developed late neonatal sepsis in the NICU and required six days of antibiotic therapy. Cranial transfontanellar ultrasound revealed sulcal narrowing and diffuse hyperechogenicity. After seven days, cranial magnetic resonance imaging (MRI) demonstrated sequelae of a severe perinatal hypoxic-ischemic event. The infant remained in the NICU for 12 days and in the ward for 10 days, being discharged with a combined breast and formula diet.
	At discharge, neurological examination showed mild generalized hypotonia, with symmetric primitive reflexes (rooting, palmar and plantar grasp, Moro, and tonic-neck reflexes). Currently, the child is three years and three months old, and follow-up evaluations by the physiotherapy team indicate normal motor development for her age.
		
	\section{Conclusion}
	The two cases described involved infants diagnosed with hypoxic-ischemic encephalopathy secondary to perinatal asphyxia who received a therapeutic hypothermia protocol for 74 hours with strict temperature monitoring. Both patients were followed in the outpatient clinic by a multidisciplinary team, and their motor development was within the normal range for age.
	These findings support the use of neonatal hypothermia as a neuroprotective intervention for infants with perinatal asphyxia, helping minimize or prevent motor developmental sequelae.
	
	\newpage
	
	% Bibliografia automática
	\printbibliography[title=References]
		
\end{document}